\documentclass[11pt,a4paper]{article}
\usepackage[utf8]{inputenc}
\usepackage{fullpage}

\usepackage{todonotes}
\usepackage[swedish]{babel}
\hyphenation{genus-balans beskriv-ning}

\title{Äskande om att lysa ut en BUL inom satsningen\\Excellence in Academy Through Gender Equality }
\author{Per Runeson and Jacek Malek, institutionen för  datavetenskap}
\begin{document}
\maketitle
%• Ämne för anställningen.
\section{Ämne för anställningen}
Vi föreslår ett biträdande universitetslektorat i datavetenskap med inriktning mot robotsystem. Inom området planerar vi att arbeta med tillämpningar inom robotkirurgi. Tjänsten föreslås tidsbegränsas till fyra år. 

%• Motivering i förhållande till LTHs strategiska plan samt framtidsvision för ämnet.
Den förslagna tjänsten bidrar i mycket hög grad till strävandena inom LTHs strategiska plan. Den:
\begin{itemize}
\item är av tvärdisciplinär karaktär, med samverkan mellan LTH och läkare på Region Skåne.
\item bidrar till bättre genusbalans inom ett ’enkönat' ämne, där obalansen långsamt vänds i) dels genom tidigare satsningar på såväl junior rekrytering som gästprofessorer, ii) dels genom att bredda tekniken åt medicinhållet, med en potential för att rekryterna fler kvinnliga studenter på sikt.
\item förenar digitaliseringsarbete med sjukvårdsutmaningar, och adresserar därmed två av de stora samhällsutmaningarna. 
\item har potential för samverkan, både för att skapa intresse för området bland barn och ungdomar, och genom industrialisering av forskningsresultat.
\end{itemize}

Vi ser tjänsten som en möjlighet för en mycket strategisk utveckling inom robotikområdet. Samarbete mellan människa och robot har beforskats i industriell miljö och i en operationssal pressas gränserna mycket längre, kombinerat med behov av AI för att realisera denna samverkan. Projektet syftar till att bland annat  undersöka det nära samarbetet mellan kirurg och robot under operationen och hur det kan ske, dvs kirurgen behöver kunna justera var roboten syr, pausa, avbryta, hålla undan tråden. Det är också intressant ur ett AI-perspektiv eftersom roboten behöver resonera och kommunicera vilken typ av stygn som passar, känna igen vävnader etc.

Tjänsten innebär också en möjlighet att bidra till en förnyelse av civilingenjörsutbildningen i Medicin och teknik, så att den får en mer systemorienterad inriktning, med kurser inom datavetenskap och robotsystem, så att de framtida civilingenjörerna kan vara med och designa digitala, medicintekniska system.

Slutligen innebär tjänsten en potiential för samverkan med det omgivande samhället. Vår huvudkandidat har vid sin tidigare anställning som doktorand vid LTH på ett enastående sätt bidragit till kunskapsspridning om robotar, genom insatser i Robotlabbet, Vattenhallen, och i populära reklamfilmer som visar robotars och robotforskningens potential\footnote{http://www.lth.se/nyheter-och-press/rsstest/visa/article/julklappsrobot-visar-vaegen-foer-framtidens-progra}. Dessutom har hon sedan disputationen varit anställd i ett start-up-företag, och därmed fått insikter i förutsättningarna för forskningsdriven innovation. 

%• Jämställdhetsredovisning för all lärarpersonal1 samt doktorander dels vid institutionen och dels vid den större enhet eller avdelning där den anställda är tänkt att verka.
\section{Jämställdhet}
Datavetenskap är ett ämne som internationellt och nationellt kämpar med genusobalans på alla nivåer, från grundstuderande till professor\footnote{Se till exempel en aktuell artikel in The Conversation https://theconversation.com/growing-role-of-artificial-intelligence-in-our-lives-is-too-important-to-leave-to-men-82708}. Vid svenska universitet och högskolor ligger andelen kvinnliga studenter i intervallet 10-15\%, och de mest framgångsrika internationella universiteten når upp till i storleksordningen 30\% kvinnliga grundutbildningsstudenter.

Institutionens lärarkår är obalanserad ut ett genusperspektiv,  se tabell nedan. 

\begin{table}[h!]
\caption{\textit{Genusfördelning för lärartjänster vid institutionen för datavetenskap resp robotgruppen. Långtidstjänstlediga lärare ej inräknade.}}
\begin{center}
\begin{tabular}{ l |c|c|c|c|}
Tjänst	&	\multicolumn{2}{|c|}{Institutionen} & \multicolumn{2}{|c|}{Forskargruppen}\\
	&	Kvinnor	& Män	&	Kvinnor	& Män	\\
\hline
Professorer	&	1&	9&	0&	2\\
Gästprofessorer & 2 & 0 & 1 & 0 \\
Universitetslektorer&	3&	13&	1&	3\\
Biträdande universitetslektorer&	2\footnotemark&	0&	0&	0\\ 
Adjunkter  &	3&	3&	0&	0\\
Postdoktorer&	0&	0&	0&	0\\
\hline
Summa	& 11 & 25 & 2 & 5\\
 & 30,6\% & 69,4\%& 28,6\%&71,4\%\\
\end{tabular}
\end{center}
\label{table:genus}
\end{table}%
\footnotetext{Varav en i slutfasen av tillsättningsförfarandet och en under prövning för befordran till universitetslektor.}

\section{Potentiella sökande}
%• Redovisning av potentiellt sökfält av underrepresenterat kön. Redovisningen ska innehålla namngivna potentiella sökande med information om lärosäte för avlagd doktorsexamen samt erfarenheter efter denna.
%• Redovisning av institutionens bedömning av de potentiella sökandes konkurrenskraft för anställningen.
%• Redovisning av de kontakter institutionen har haft med de potentiella sökande avseende deras intresse för anställningen.
Som tidigare redovisats i samband med rekrytering av professor och gästprofessor inom ämnet är det internationellt ett oerhört få kvinnliga potentiella sökande på professorsnivå och detsamma gäller på universitetslektorsnivå. Vid en tidigare rekrytering, till en universitetslektorstjänst med bredare ämnesbeskrivning inom programvaruteknik, inkom 37 ansökningar, varav endast 3 från kvinnor (8\%). Tyvärr var samtliga dessa tre utanför ämnesområdet. Utlysningen av doktorandtjänst i datavetenskap med inriktning mot robotteknik fick nyligen 54 ansökningar, och en av de 9 sökande kvinnorna erbjöds tjänsten (17\%!), men hon tackade tyvärr nej pga erbjudande från andra universitet, och utlysningen avbröts. Det är alltså ett svårt område att överhuvudtaget hitta kvinnliga sökande. 

Därför är det en fantastisk möjlighet för institutionen och LTH, att \textbf{Dr. Maj Stenmark} har uttalat intresse för att söka ett biträdande universitetslektorat om ett sådant utlyses. Maj disputerade vid institutionen för datavetenskap våren 2017 med en avhandling om ``Intuitive Instruction of Industrial Robots''. Sedan dess har hon arbetat i startup-företaget Cognibotics på Ideon. 

Maj har, givet sin tidiga position i karriären, en omfattande publikationslista. \todo{Jacek, mera om Majs meriter?} Sammanvägt med hennes mycket goda pedagogiska meriter och erfarenheter av samverkansverksamheter framstår hennes konkurrenskraft för anställningen som mycket god.

\section{Finansiering}
%• Redovisning över institutionens motfinansiering.
Institutionen står mitt i en större nysatsning på robotgruppen, som består i en nyrekryterad professor med tillhörande doktorander och postdoc som finansieras av WASP, samt en gästprofessor i Hedda Anderssons namn. Institutionen får startbidrag till professorn samt strategiskt stöd till robotlabbet tillsammans med reglerteknik. Detta medför att institutionen har fått upp volymen igen och fått en kritisk massa i robotgruppen. 

Den del av kostnaden för det biträdande universitetslektoratet som inte täcks av de centrala medlen, täcks dels av grunddutbildningsmedel och dels av de ovan nämnda start- och strategibidragen. 


%• Den önskade anställningstiden ska anges. Längre anställningstid än fyra år ska motiveras.

%Majs text: Det första jag undrar är om det är "fritt fram" att formulera forskningsprojekt? Hittills har jag, Anders och kirurgen Kiet Tran på Region Skåne "skunkjobbat" för att göra YuMi till en Syborg och nu när roboten har ett gripdon och kan sy i initiala experiment så hade det varit intressant att ta det projektet vidare. Tyvärr passar det inte riktigt Cognibotics område eftersom det är alldeles för forskningsinriktat. Kiet håller på att ansöka om anslag för att få en robotiserad operationssal (vilket naturligtvis inte heller är klart än) och Kiet och jag har funderat på hur vi kan få till en postdoc till mig inom robotkirurgi, så detta kan vara ett sätt att få igång det ... :P Det finns flera intressanta forskningsområden relaterat till robotkirurgin, bland annat att undersöka det nära samarbetet mellan kirurg och robot under operationen och hur det kan ske, dvs kirurgen behöver kunna justera var roboten syr, pausa, avbryta, hålla undan tråden, men det är också intressant ur ett AI perspektiv eftersom roboten behöver resonera och kommunicera vilken typ av stygn som passar, känna igen vävnader etc ...

\end{document}